\documentclass[letter,12pt]{article}

\usepackage[T1]{fontenc}
\usepackage{lmodern}
\renewcommand*\familydefault{\sfdefault} 
\usepackage[spanish]{babel}
\usepackage[utf8]{inputenc}
\usepackage[pdftex]{graphicx}
\usepackage{hyperref}
\usepackage{fancyhdr}
\usepackage{lastpage}
\usepackage{tabularx}
\usepackage{float}
\pagestyle{fancy} 
\fancypagestyle{plain}
{%
         \fancyhead[l]{}
         \fancyhead[r]{}
         \fancyhead[c]{}
         \renewcommand{\headrulewidth}{0.5pt}
         \fancyfoot[l]{SCESI \\ Sociedad Cient\'ifica de Estudiantes de Sistemas e Inform\'atica}
         \fancyfoot[c]{}
         %\fancyfoot[r]{\thepage/\pageref{LastPage}}
         \renewcommand{\footrulewidth}{0.5pt}
}
% Para el resto de páginas
\lhead{Proyecto Seguridad de jornadas}
\chead{}
\rhead{\includegraphics[width=0.1\textwidth]{logo2.png}}
\renewcommand{\headrulewidth}{0.4pt}
\lfoot{SCESI \\ Sociedad Cient\'ifica de Estudiantes de Sistemas e Inform\'atica \\ 
\url {http://scesi.fcyt.umss.edu.bo}}
\cfoot{}
\rfoot{\thepage/\pageref{LastPage}}
\renewcommand{\footrulewidth}{0.4pt}
% Title Page
\title{Seguridad de Jornadas}
%NOMBRE DEL AUTOR O AUTORES
\author{
	Elvis Ramirez 			\\ 
	Fernando Solis Tapia	\\ 
	Jorge Lipa Challapa		\\ 
	David Mercado Arispe	\\ 
	Juan Ach\'a Maraza		\\ 
	Jose Valdivia			\\ 
	Valerio Terrazas
}

\begin{document}
\maketitle
\begin{center}
    \includegraphics[width=1.0\textwidth]{logo.png} 
\end{center}
\begin{center}
    \url {http://scesi.fcyt.umss.edu.bo}
\end{center}
\pagebreak
\tableofcontents
\pagebreak
%\begin{abstract}

%\end{abstract}
%\pagebreak
\section{Introducci\'on}

Uno de los problemas de seguridad mas frecuentes a sido el control acceso a los ambientes que requieren un acceso especial. Las diferentes soluciones que se dan para estos problemas llegan a tener dificultades, ya sea por tiempo, recursos o personal administrativo, los costos de estos sistemas llegan a ser muy altos o tiene una implementacion muy confusa para personas normales. \\

El sistema \textit{Seguridad de jornadas}, como aplicaci\'on permitira el control de de acceso y demas actividades que necesiten una verificaci\'on de itentidad utlizando amarino, tecnologia nos permitira elaborar instrumentos permitiendo el acceso controlado de los participantes de manera r\'apida y eficiente con una implementacion simple y directa. \textit{Amarino} es la combinaci\'on de tecnologias, en este caso Android ( encontrada en dispositivos moviles )y Arduino (posee variedad en diferentes modulos con los que trabaja).

\section{Antecedentes}

RFID \textit{( Identificaci\'on de radio frecuencia)} es usado por varios dispositivos de identificaci\'on como art\'iculos comerciales o identificadores personales de esta manera se permite un control muy estricto sobre las actividades que giran en torno a estos. Al ser un complemento muy requerido arduino a dejado a disposition modulos compatibles con las placas Arduino Uno, Mega, Leopard, Nano, dando asi un f\'acil acceso a los desarrolladores para implementaciones mas sofisticadas y simples de utilizar.\\

Unas de las herramientas mas cotidianas que utilizan personas normales son los  dispositivos moviles que bajo una infraestructura android ofrecen servicios de conectividad para diferentes redes y soporte de aplicaciones para un sinf\'in de actividades, un sistemas que esta disponible para dispositivosde alto y bajor rendimiento.

 
\section{Definici\'on del Problema}

Uno de los problemas mas regulares en las jornadas a sido el control de asistencia de los participantes y diferentes actividades que requieren de identificacion para una o varias operaciones tal es el caso, antes de entrar a un evento al que se a registrado, saber si asistio llega a ser muy problematico para el personal organizador pues requiere de mucho esfuerzo. \\

Dadas estas circunstancias el saber si personas registradas o no han ingresaso al ambiente o han realizado alguna actividad que esta limitada a un grupo personas, esto lleva a variaciones que tiene que ser corregidas en ese momento para evitar problemas mayores.\\

Los metodos de recolecci\'on de informaci\'on sobre los participante, llega a ser muy amplia lo que ocasiona que el procesamiento de documentos posteriores al evento sea moroso y repetitivo al tratar de verificar la correctitud de la informacion proporsionada, no olvidemos mencionar la generacion de reportes sobre las actividades o procesos que se realizaron. Esta informaci\'on es muy util para planificaciones futuras por el flujo de verificacion de todas las actividades y procesos sobre los eventos  esto mostrara de manera detallada mas a fondo si se realizaron de manera adecuada o falto personal de control y demas aspectos personales.

\section{Objetivo General}
	\begin{itemize}
		\item Asegurar el control de las actividades o procesos que esten destinadas a un grupo de personas autentificadas mediantes tarjetas de identificaci\'on unicas, utilizando dispositivos electronicos para su validaci\'on.
	\end{itemize}

\section{Objetivos Espec\'ificos}
	\begin{itemize}
	
		\item Establecer un componente electr/'onico que permita verificar el estado de registro del participante.
		\item Controlar de forma remota el acceso de los participantes
		\item Facilitar la generacion de reportes sobre la asistencia de los participantes.
		\item Simplificar la generaci\'on de certificados de los participantes y personal administrativo. 
	\end{itemize}

\section{Recursos}

Para la realizacion de este proyecto contaremos dispositivos electronicos y moviles para la ejecucion, se especificara en base al prototipo que se pretende realizar:\\

\pagebreak

\begin{table}
	\centering
	\begin{tabularx}{15cm}{X|X}
	Dispositivo electr\'onico 					& Descripci\'on									\\ 
	\hline
	Arduino Uno 								& Placa electr\'onica de hardware libre 
												  para la creaci\'on de prototipos basada en 
												  software y hardware libre                     \\
												& \\  
	Tarjeta RFID Mifare S50 1k 13.56MHz         & Tarjetas plásticas que incorporan una 
												  tecnología de comunicaci\'on radiofrecuencia  \\
												& \\
	Arduino RS232 13.56MHZ RFID 				& Modulo de lectura y escritura de
												  tarjejas complatibles con Mifare S50  		\\
												& \\												
	Bluetooth HC-06 serial 4 Pin 				& El modulo BlueTooth HC-06 utiliza el protocolo 
												  UART RS 232 serial. Es ideal para 
												  aplicaciones inal\'ambricas, f\'acil de  
												  implementar en componentes Arduino			\\
												& \\
	\end{tabularx} 
	\caption{Tabla de dispositivos electronicos}		
\end{table}	

\section{Herramientas}
Para cunplir con el proposito de este proyecto, describiremos en detalle las herramientas que desarrollo y entornos de trabajo bajo los cuales se 			elaborar el proyecto. Cabe resaltar que estas herramientas mencionadas no son aplicables a futuras actualizaciones.
	
	\subsection{Herramientas de desarrollo}
	
	Esta seccion detalla el software que se utilizara para su desarrollo en todos los ambientes que cubrira el proyecto y detallar su descricion con la versi\'on y y su desempenio segun \textbf{Tabla de herramientas de desarrollo}\ref{herramienta_desarrollo}.
	

	\begin{table}[H]
		\begin{tabular}{l|c|l}
			Herramienta       & Versi\'on   & Descripci\'on					   \\
			\hline
			Apache 			  & 2.2			& Servidor de aplicaciones web     \\
			Php				  & 5.3			& Lenguage de programacion web     \\
			MySQL             & 5.2			& Servidor de base de datos        \\
			Yii framework     & 1.1.13  	& Entorno de trabajo basado en php \\
			Android           & 2.3.3		& Entorno de trabajo basado en java\\
			Arduino 		  & 1.0.5		& Entorno de trabajo basado en     \\
							  &             & Processing					   \\	
		\end{tabular}
		\caption{ Tabla de herramientas de desarrollo}
		\label{herramienta_desarrollo}
	\end{table}

\section{Metodo o Técnica}

El metodo de desarrollo que se tomara se realizara, sera un derivado de las metodologias agiles (SCRUM) que estara detallada en los ciclos de desarrollo. Primeramente  detallaremos el personal o usuarios finales que manejaran el sistema y posteriormente .

	\subsection{ Usuarios finales }
	
	\begin{description}
		
		\item [Participante del evento] Persona que portara la identificaci\'on para participar en la actividad.
		\item [Encargado de registro de participantes] Persona que registra la tarjeta con la identificacion e informacion del participante.
		\item [Encargado del control de las actividades]Esta persona controla el ingreso a la actividad, comprobando la actividad que se desea realizar.  extrayendo la informacion basica, el personal del control de actividades puede variar dependiendo de cantidad de las actividades que se tengan planeadas.
		\item [Administrador de actividades] Persona que se encarga de administrar la informacion que se genera a partir de las verificaciones en las actividades.
	\end{description}
	
	\subsection{Equipamiento}
	
	\begin{description}
		\item[Tarjeta de identificaci\'on] Tarjeta que utiliza el participante para identificarse.
		\item[Lector / escritor RFID] Dispositovo que facilita la lectura y escritura de las tarjetas de identificaci\'on.
		\item[Aplicaci\'on movil - Notificaciones] Aplicacion que recibe notificaci\'ones sobre las lecturas que se haga en el lector RFID
		\item[Aplicaci\'on movil - Verificaci\'on actividad] Aplicaci\o'n que permite validar la entreda de las identificaciones.
		\item[Aplicaci\'on movil - Registro de usuarios] Aplicacion que permite el registro de nuevos participantes.
	\end{description}		
	
\section{Justificaci\'on}

Tener el acceso controlable de todos los participntes, con datos precisos referentes a horas de entrada y salida, asistencia, generacion de reportes sobre 
la concurrencia a las actividades, datos estadisticos sobre el desarrollo del evento.\\

Con el sistema de Seguridad en jornadas tendremos a disposicion una manera simple de control, que los mismos participantes encontrarar mas rapido y eficiente acortando asi la concurrencia, verificaciones. De este modo el trabajo del personal administrativo quedara reducido a que este presente el participante.

\end{document}          